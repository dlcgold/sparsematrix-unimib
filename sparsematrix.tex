\documentclass[a4paper,12pt, oneside]{article}
% \usepackage{fullpage}
\usepackage[italian]{babel}
\usepackage[utf8]{inputenc}
\usepackage{amssymb}
\usepackage{amsthm}
\usepackage{graphics}
\usepackage{amsfonts}
\usepackage{listings}
\usepackage{amsmath}
\usepackage{amstext}
\usepackage{engrec}
\usepackage{rotating}
\usepackage[safe,extra]{tipa}
\usepackage{showkeys}
\usepackage{multirow}
\usepackage{hyperref}
\usepackage{sectsty}
\usepackage{mathtools}
\usepackage{microtype}
\usepackage{enumerate}
\usepackage{braket}
\usepackage{marginnote}
\usepackage{pgfplots}
\usepackage{cancel}
\usepackage{polynom}
\usepackage{booktabs}
\usepackage{enumitem}
\usepackage{framed}
\usepackage{algorithm}
\usepackage{algpseudocode}
\usepackage{pdfpages}
\usepackage{pgfplots}
\usepackage[cache=false]{minted}

\usepackage[usenames,dvipsnames]{pstricks}
\usepackage{epsfig}
\usepackage{pst-grad} % For gradients
\usepackage{pst-plot} % For axes
\usepackage[space]{grffile} % For spaces in paths
\usepackage{etoolbox} % For spaces in paths
\makeatletter % For spaces in paths
\patchcmd\Gread@eps{\@inputcheck#1 }{\@inputcheck"#1"\relax}{}{}
\makeatother

\usepackage{tikz}\usetikzlibrary{er}\tikzset{multi  attribute /.style={attribute ,double  distance =1.5pt}}\tikzset{derived  attribute /.style={attribute ,dashed}}\tikzset{total /.style={double  distance =1.5pt}}\tikzset{every  entity /.style={draw=orange , fill=orange!20}}\tikzset{every  attribute /.style={draw=MediumPurple1, fill=MediumPurple1!20}}\tikzset{every  relationship /.style={draw=Chartreuse2, fill=Chartreuse2!20}}\newcommand{\key}[1]{\underline{#1}}

%\usepackage{fancyhdr}

\usepackage{tikz}
\usetikzlibrary{automata,positioning}


\title{Relazione Progetto\\
  Programmazione C++\\
  \large Sparse Matrix}

\author{Davide Cozzi\\
  829827\\
  \href{mailto:d.cozzi@campus.unimib.it}{d.cozzi@campus.unimib.it}}
\date{}

\pgfplotsset{compat=1.13}
\begin{document}
\maketitle

\definecolor{shadecolor}{gray}{0.80}
\setlist{leftmargin = 2cm}
\newtheorem{teorema}{Teorema}
\newtheorem{definizione}{Definizione}
\newtheorem{esempio}{Esempio}
\newtheorem{corollario}{Corollario}
\newtheorem{lemma}{Lemma}
\newtheorem{osservazione}{Osservazione}
\newtheorem{nota}{Nota}
\newtheorem{esercizio}{Esercizio}

\renewcommand{\chaptermark}[1]{%
  \markboth{\chaptername
    \ \thechapter.\ #1}{}}
\renewcommand{\sectionmark}[1]{\markright{\thesection.\ #1}}
\allsectionsfont{\centering}

\section*{La Struttura Dati}
Ragionando sulla consegna del progetto ho concluso che il modo
migliore per implementare una \textbf{sparse matrix} fosse quello di
utilizzare una variante della \textit{linked list}, dove gli elementi vengono
caricati secondo l'ordine di due indici (che rappresentano la
posizione di una cella in una matrice di implementazione
classica). L'uso di questa variante della \textit{linked list} permette di
salvare in memoria solamente i valori indicati dall'utente (ognuno
nella posizione desiderata).\\
In questa implementazione ogni \textttt{node} della \textbf{sparse matrix}
contiene:
\begin{itemize}
  \item un puntatore, privato, al \textttt{node} successivo. La scelta
  di definire questo attributo \texttt{private} è causata dal fatto
  che l'utente non deve essere conscio della struttura dati in uso
  \item una \texttt{struct} \texttt{element} contenente:
  \begin{itemize}

  \item il valore che deve essere salvato
  \item i due indici mediante i quali, logicamente, accedere al
  valore. I due indici sono definiti \texttt{const} in quanto non
  modificabili dopo la creazione della cella
  
  \end{itemize}
\end{itemize}
Per questa \textit{sottoclasse} \texttt{node} viene implementato un
\textit{costruttore} che genera un node a partire dal valore che si vuole
aggiungere e dalla posizione in cui lo si vorrebbe aggiungere, senza
specificare (e impostandolo di default a \texttt{nullptr}) il
\texttt{node} successivo, e, ovviamente, viene implementato il
\textit{distruttore}, che si limita a settare a \texttt{nullptr} il
nodo successivo (questo perché la distruzione dell'intera struttura
dati viene implementata nel \textit{distruttore} di
\texttt{sparse\_matrix}).\\
Parliamo ora della struttura dati completa, chiamata
\texttt{sparse\_matrix}. Si hanno una serie di attributi privati
necessari per il corretto funzionamento della stessa:
\begin{itemize}
  \item il valore di default scelto obbligatoriamente dall'utente al
  momento della creazione di una matrice sparsa
  \item un booleano per verificare se la matrice è stata definita
  dall'utente con una dimensione predefinita
  \item indicazioni riguardanti il numero di righe e colonne
  \item il nodo \texttt{head} che punta all'inizio della struttura
  dati
  \item il valore di elementi effettivamente contenuti nella struttura dati
\end{itemize}
Si hanno quindi diversi costruttori utili:
\begin{itemize}
  \item si ha innanzitutto il costruttore base con solo l'indicazione
  di default, senza quindi indicazione delle dimensioni della matrice
  (che diventa, a livello teorico, una matrice quadrata di massima
  dimensione intera al quadrato)
  \item il costruttore di una matrice sparsa di dimensione teorica
  definita dall'utente
  \item costruttore copia a partire da un'altra
  \texttt{sparse\_matrix} definita sullo stesso tipo
  \item costruttore copia a partire da un'altra
  \texttt{sparse\_matrix} definita su un altro tipo (con controllo del
  cast delegato, secondo specifiche, al compilatore)
\end{itemize}
Si hanno poi, ovviamente, le implementazioni dei metodi
\textit{getter}.\\
Per quanto riguarda il distruttore si ha che esso chiama un metodo
pubblico \texttt{clear()} che si appoggia ad un metodo privato
\texttt{recursive\_clear(head)}, che, partendo appunto dal primo
elemento, setta a \texttt{nullptr} i nodi raggiungibili da
\texttt{head}, settando, infine, anche \texttt{head} a
\texttt{nullptr}. Il metodo \texttt{clear()} è pubblico in quanto
permette all'utente di eliminare una \texttt{sparse\_matrix} da lui
creata, evitando quindi di incorrere in \textit{memory leaks}.
\newpage
\section*{Funzioni Principali}
Ovviamente il primo discorso da affrontare è l'aggiunta di un
determinato elemento in una certa posizione della nostra matrice
sparsa, che viene gestito da un metodo \texttt{add(valore,
  posizione\_x, posizione\_y)}. Si hanno diverse possibili situazioni
per l'inserimento di un nuovo valore:
\begin{itemize}
  \item il caso ``banale'' è quello in cui ancora si ha una matrice
  sparsa senza alcun elemento, in tal caso si genera un nuovo
  \texttt{node} costruito sui dati in input e tale \texttt{node}
  diventa la nuova \texttt{head} della matrice sparsa
  \item si ha poi il caso in cui la \texttt{head} attuale è
  posizionata in una cella logicamente successiva a quella
  dell'elemento che si vuole inserire. Questa situazione si puà
  verificare in 2 casi:
  \begin{enumerate}
    \item l'\texttt{head} attuale si trova sulla stessa riga teorica
    dell'elemento che si vuole aggiungere ma ad una colonna successiva
    \item l'\texttt{head} attuale si trova ad una riga successiva
  \end{enumerate}
  \item se non si rientra nelle due casistiche precedenti si itera su
  tutta la matrice sparsa, fino all'avvenuta di una delle due
  possibili condizioni di arresto:
  \begin{enumerate}
    \item si è arrivato all'ultimo elemento della matrice sparsa (che
    quindi punta a \texttt{nullptr} come nodo successivo). In tal
    caso, se gli indici del nodo che si vuole inserire coincidono con
    quelli dell'ultimo elemento se ne sovrascrive il valore,
    ricordandosi di cancellare il nodo che si voleva inserire per
    evitare \textit{memory leaks}, altrimenti lo si inserisce come
    ultimo elemento (il vecchio nodo ``tail'' ora punterà a tale
    elemento che a sua volta punterà a \texttt{nullptr})
    \item si sta valutando l'unico caso possibile restante, ovvero
    l'elemento che vogliamo inserire viene aggiunto in mezzo alla
    matrice sparsa, nella giusta posizione ordinata sui due indici. In
    tal caso, se gli indici del nodo che si vuole inserire coincidono
    con quelli dell'ultimo elemento se ne sovrascrive il valore,
    ricordandosi di cancellare il nodo che si voleva inserire per
    evitare \textit{memory leaks}, altrimenti inserisco l'elemento che
    si vuole aggiungere in mezzo alla struttura dati, nel punto
    corretto, sistemando i puntatori ai nodi successivi dell'elemento
    che si vuole inserire e del suo precedente
  \end{enumerate}
\end{itemize}
lo stesso metodo aggiorna anche il numero massimo di righe e colonne
della matrice ipotetica, cercando il massimo indice di riga e di
colonna.
\\
L'eventuale inserimento di valori fuori indice viene gestito tramite
un'eccezione \texttt{IndexOutOfBoundsException()}.\\
Sempre da specifica si ha poi l'implementazione del supporto agli
iteratori di tipo forward in e scrittura. Questo viene implementato
con le due classi \texttt{iterator} e \texttt{const\_iterator} che
vengono costruiti a partire da \texttt{node}. Si ha quindi la
definizione dei metodi \texttt{begin()}, che restituisce un
\texttt{iterator} o un \texttt{const\_iterator} costruiti a partire da
\texttt{head}, e \texttt{end()}, che restituisce un
\texttt{iterator} o un \texttt{const\_iterator} costruiti a partire da
\texttt{nullptr}, per segnalare la fine della matrice sparsa. \\
Un iteratore restituisce in lettura valore e posizione, in particolare
\texttt{iterator} permette l'accesso in scrittura (per la modifica)
del valore di una cella ma non degli indici della stessa.\\
Per l'accesso in lettura dei valori della matrice si è anche
implementato l'overload di \texttt{operator()}, che permette l'accesso
ad un determinato valore di una cella mediante i due indici. Dato che
da specifica deve avvenire la restituzione del valore di default, nel
caso in cui si chieda il valore di una cella non indicizzata nella
nostra matrice sparsa, si procede iterando nella matrice sparsa come
nel caso della \texttt{add} fino ad ottenere una condizione d'arresto e, se
si incontra il \texttt{node} con indici uguali a quelli richiesti in
uscita dal ciclo se ne restituisce il valore, altrimenti si
restituisce il valore di default.\\
Infine viene implementata una funzione gloable
\texttt{evaluate(sparse\_matrix, predicato)} che conta il numero di
elementi nella matrice che soddisfano il predicato (considerando anche
i valori di default). Si itera quindi sulla matrice sparsa contando
tale numero. Infine si verifica se il valore di default lo soddisfa e,
in caso positivo, si conta il numero di celle con valore di
default (ovvero numero totale di celle meno il numero elementi
effettivamente allocati, numero che è salvato nell'attributo
\texttt{size}) e lo si somma al conteggio sopra effettuato,
restituendo il risultato.
\newpage
\section*{Test}
Il primo test viene effettuando generando una \texttt{sparse\_matrix}
con valori di tipo \texttt{float} di dimensione definita $3\times
3$. Vengono testati metodi che ritornano il numero di righe e di
colonne. Si usa poi un metodo \texttt{printdef} per la stampa di
dimensioni predefinite (possibilmente piccole) che itera come se si
avesse a che fare con una matrice standard, stampando sia gli elementi
caricati dall'utente che gli eventuali valori di default. Si procede
poi con il test della funzione globale generica \texttt{evaluate} con
il controllo del risultato mediante \texttt{assert}.\\
Con la stessa tecnica si verifica anche la buona riuscita della
costruzione di una nuova \texttt{sparse\_matrix} mediante
\textit{cast}. Viene verificato anche l'assegnamento tra due
\texttt{sparse\_matrix} e l'assegnamento tra iteratori.\\
Vengono poi effettuati test simili su una matrice, di dimensioni non
specificate, dove non vengono però aggiunti valori e su una
\texttt{sparse\_matrix} definita su un tipo custom \texttt{point}.
\end{document}
